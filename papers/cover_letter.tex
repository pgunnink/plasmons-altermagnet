\documentclass{letter}
\usepackage{geometry}
\geometry{
	paper=a4paper,
	margin=100pt,
}
\usepackage{amsmath}
\usepackage{bm}
\usepackage{hyperref}
\usepackage{txfonts}
\signature{Pieter Gunnink, Jairo Sinova, Alexander Mook}
\address{Pieter Gunnink \\ Institute of Physics \\
	Johannes Gutenberg University Mainz \\
	55099 Mainz \\ Germany
}
\begin{document}
	
	\begin{letter}{}
		\opening{Dear editors,}
		
		
		We are pleased to submit our manuscript entitled ``Spin demons in $d$-wave altermagnets'' for publication in Physical Review Letters.
		
		
		The existence of a \emph{demon} was first proposed by Pines [Canadian Journal of Physics \textbf{34}, 1379 (1956)], for two electron bands with heavy and light effective electron masses. Demons are formed by the out-of-phase movement of electron densities in different bands, and are acoustic and electrically neutral. They therefore do not couple to light, which has hindered their direct detection in a three-dimensional metal. Their existence was first shown by Husain \textit{et al.} [Nature 621, \textbf{66} (2023)] through momentum resolved electron energy-loss spectroscopy (EELS). 
		
		In this work, we demonstrate that the spin densities in a $d$-wave altermagnet can also oscillate out-of-phase, due to the altermagnetic spin-split band structure.  We show that this realizes a spin-polarized version of the demon first proposed by Pines---which we thus call a \emph{spin demon}. We demonstrate that even though the spin demon is not undamped (in contrast to a conventional charge plasmon), it has a high quality factor and remains thus well defined.
		Uniquely for altermagnets, we demonstrate that the spin demon inherits the altermagnetic symmetry, by calculating its magnetic moment, which switches sign as the spin demon is rotated in the altermagnetic spin-split plane.
		
		
		In contrast to the conventional demon, the spin demon has a spin polarization and can thus be potentially easier to detect. In particular, the spin demon has a strong signature in the spin-spin response function, $\chi_{S_zS_z}(\bm q,\omega)$, which can be directly probed through
		spin-polarized electron energy-loss spectroscopy (SPEELS). 
		
		Since our finding of a spin demon expands the landscape of collective spin phenomena and offers a clear route toward its observation in real materials, we hope that you share our enthusiasm and are looking forward to hearing from you.
		
		\closing{Yours sincerely,}
		
		
		
		
	\end{letter}
\end{document}


We report the theoretical discovery of a novel quasiparticle in altermagnets: a neutral, spin-density collective mode that we term a spin demon. We show that this excitation is underdamped in two- and three-dimensional d-wave altermagnets, making it accessible to experimental detection. The spin demon carries a magnetic moment that reflects the underlying d-wave symmetry of the host material. Our findings identify a fundamentally new type of spin carrier in unconventional magnets, expanding the landscape of collective spin phenomena and offering a clear route toward its observation in real materials.
